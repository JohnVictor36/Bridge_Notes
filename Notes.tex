\documentclass[UTF8]{article}
\usepackage[UTF8]{ctex}
\title{Bridge Notes}
\author{JohnVictor}
\date{}
\begin{document}
\maketitle
\section{建设性叫牌}
\subsection*{1H-1NT-2D-?}

此处 2NT 就是 10-12 邀请 3NT

2H 为两张弱牌;4D+2H 一般选择 2H,避免打 4-3

3C 为弱牌修正定约,8-10,6 张好套,极限情况可以是不好的 11 点

\subsection*{1H-1NT-2C-?}

2D 修正定约,建议有六张,除非是 5521 的五张

2S 是对 C 支持的邀请,有 4 张 C

3C 建设性,8-10,目的是让叫牌不结束,同伴中限以上可以继续叫牌,一般选择 3NT,畸形选择 5C

3D 六张 D 好套,邀请 3NT

3H 是 10-11,3+H,1NT 过渡邀请 4H

3S 为 S 单缺,五张 C,邀请 5C

3NT 为赌博性,一般为 C 上的好配合,同伴无法赌博(例如 55 套,10p)可以改 4C

畸形牌可以 4C 邀请 5C,例如 6+C

4H 是看见双套配合(4+C)直接叫,10-11

\subsection*{开叫人的再叫}

如果看见非邀请的叫牌:低限一律 pass,中限没有额外长度一律最低水平 2NT/3NT,额外长度在三阶表示

如果看见 2NT: 3C 为低限 55,3H 为非低限 64,低限 64 pass

\subsection*{1S-1NT-2H}

保四个 H

\subsection*{1S-1NT-2H-2NT-?}

pass 没有额外长度,低限

3NT 没有额外长度,非低限

3C 低限 55

3D 低限 64

3H 非低限 55

3S 非低限 64

这里低限表示最低限,也就是 12-13

\subsection*{1D-1H-1S-2C-2D-2NT-?}

这里有三个 H 非低限一定要表示出来,叫 3H;一般地,是未表示的高花中比较长的,一定要表示,避免丢失 4M

\subsection*{1D-1H-1S-2C-?}

中限以上不想打 2D 可以直接越过 2D,2H 为三个 H,2NT 为 C 有止张,2S 为 C 没有止张

\subsection*{1D-1H-2H-2NT-?}

3C 是低限 4C5D3H1S,3D 是低限 5422,3H 是低限四个 H 均型,3S 是高限 S 单缺,4C 是高限 C 单缺,4H 高限无单缺或者低限有单缺
\end{document}